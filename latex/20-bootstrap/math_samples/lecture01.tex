% This is a simple LaTex sample document that gives a submission format
%   for IEEE PAMI-TC conference submissions.  Use at your own risk.

% Make two column format for LaTex 2e.\
\documentclass[11pt,twocolumn]{article} %,twocolumn

%\usepackage{times,amsmath,amsfonts}

% Use following instead for LaTex 2.09 (may need some other mods as well).
%\documentstyle[times,twocolumn]{article}
\usepackage[dvips]{graphicx,graphics}
% Set dimensions of columns, gap between columns, and paragraph indent
\setlength{\textheight}{9.2in} \setlength{\textwidth}{6.875in}
%\setlength{\columnsep}{0.3125in} \setlength{\topmargin}{0in}
\setlength{\headheight}{0in} \setlength{\headsep}{-1in}
\setlength{\parindent}{1pc}
\setlength{\oddsidemargin}{-.1875in}  % Centers text.
\setlength{\evensidemargin}{-.1875in}

% Add the period after section numbers.  Adjust spacing.
\newcommand{\Section}[1]{\vspace{-8pt}\section{\hskip -1em.~~#1}\vspace{-3pt}}
\newcommand{\SubSection}[1]{\vspace{-3pt}\subsection{\hskip -1em.~~#1}
        \vspace{-3pt}}
\newcommand{\bqn}{\begin{eqnarray}}
\newcommand{\eqn}{\end{eqnarray}}
\newcommand {\diff}[1] {\frac{\partial}{\partial #1}}
\newcommand{\jacob}[3]{\frac{\partial^2 #3}{\partial #1 \partial #2}}
\newcommand{\der}[2]{\frac{\partial #2}{\partial #1}}
\begin{document}

% Make title bold and 14 pt font (Latex default is non-bold, 16pt)
\title{Stat 471: Lecture 1\\
MATLAB/OCTAVE}
% For single author (just remove % characters)
\author{Moo K. Chung\\
mchung@stat.wisc.edu}
% For two authors (default example)
\maketitle \thispagestyle{empty}
\begin{enumerate}
\item Statistics is concerned with the science of uncertainty.
Computational statistics is a collection of techniques that have a
strong focus on the exploitation of computing in the creation of
new statistical methodology (Wegman, 1988).

\item Cheap computation enabled us to store and process massive
data so that we often collect the data first and then design study
to gain information. The data sets these days is very large and
high-dimensional.

\item Data types: scalar, vector, matrix, tensor.\\
{\em Observed value $=$ true value $+$ measurement error}.
$$Y_{ijk} = \mu + e_{ijk},$$
where $ijk$ is in certain index set. For instance, if $Y_i, i \in
\mathbb{R}$, $Y_i$ is a continuous observation (function) such as
temperature. In the case of high dimensional index set, $Y_i \in
\mathbb{R}^n$, $Y_i$ becomes $n$-dimensional data. Tensorial is a
generalization of matrix data. Think a tensor as a high
dimensional array with complicated relationship among elements.
For data with the large size of index set, computation-intensive
statistical methods (Efron and Tibshirani, 1991) are more
practical than the traditional design oriented data analysis.

\item Computer languages: MATLAB or Octave will be used as the
language of instruction. GNU Octave is a free version of MATLAB
(www.octave.org) developed by the department of chemical
engineering, WISC. Splus or R are programming language mainly used
by statisticians. It shares similar language structure with LISP.
If you don't have access to any computer or programming language,
create one in the statistics computer lab located in the first
floor. Read first http://www.cs.wisc.edu/csl/doc\\
/faq/started/index.html.

\item Type either octave or {\em matlab} from the {\em linux}
prompt to run the programs. For octave, you may possibly use text
editor such as Emacs to save your work.

\item Uniform random numbers $U \sim Unif(0,1)$. Using $U$,
generate 1000 uniform random numbers $X \sim Unif(-1,5)$.
\begin{verbatim}
>> u = rand(1000,1);
>> size(u)
ans =
        1000           1
>> x = 4*u-1;
>> rand
ans =
    0.4048
>> rand
ans =
    0.6271
>> s=rand('state')
s =
    0.8598
    0.1398
    ......
>> rand('state',s)
>> rand
ans =
    0.4048
\end{verbatim}

Use help command to see the grammar. ex. {\tt help rand}.

\end{enumerate}

\end{document}
