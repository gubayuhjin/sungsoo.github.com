% This is a simple LaTex sample document that gives a submission format
%   for IEEE PAMI-TC conference submissions.  Use at your own risk.

% Make two column format for LaTex 2e.\
%\documentclass[10pt,twocolumn]{article} %,twocolumn
\documentclass[10pt,twocolumn]{article} %,twocolumn
\usepackage{times,amsmath,amsfonts}
\usepackage[dvips]{graphicx,graphics}
% Use following instead for LaTex 2.09 (may need some other mods as well).
%\documentstyle[times,twocolumn]{article}

% Set dimensions of columns, gap between columns, and paragraph indent

%\setlength{\columnsep}{0.3125in} \setlength{\topmargin}{0in}

% Add the period after section numbers.  Adjust spacing.

\setlength{\textheight}{10.5in} \setlength{\textwidth}{7.3in}
%\setlength{\columnsep}{0.3125in} \setlength{\topmargin}{0in}
\setlength{\headheight}{0in} \setlength{\headsep}{-1in}
\setlength{\parindent}{1pc}
\setlength{\oddsidemargin}{-0.6in}  % Centers text.
\setlength{\evensidemargin}{-0.6in}

% Add the period after section numbers.  Adjust spacing.
\newcommand{\Section}[1]{\vspace{-5pt}\section{\hskip -1em.~~#1}\vspace{-3pt}}
\newcommand{\SubSection}[1]{\vspace{-2pt}\subsection{\hskip -1em.~~#1}
        \vspace{-3pt}}
\newcommand{\bqn}{\begin{eqnarray*}}
\newcommand{\eqn}{\end{eqnarray*}}
\newcommand{\bq}{\begin{eqnarray}}
\newcommand{\eq}{\end{eqnarray}}


\begin{document}

% Make title bold and 14 pt font (Latex default is non-bold, 16pt)
\title{Stat 471: Lecture 34\\
Gaussian Processes II.}
% For single author (just remove % characters)
\author{Moo K. Chung \tt{mchung@stat.wisc.edu}}
% For two authors (default example)
\maketitle \thispagestyle{empty}


%\begin{figure}
%\centering
%\renewcommand{\baselinestretch}{1}
%\includegraphics[scale=0.8]{homework05-1.eps}
%\end{figure}

\begin{enumerate}

\item {\em White noise} is defined as a random process whose
covariance function is proportional to the Dirac-delta function
$\delta$, i.e. $R(t,s) \propto \delta(t-s)$. We can define it as
the limiting density function of $N(0,\sigma^2I_N)$ as $\sigma \to
0$. %One property of the Dirac-delta is
%$$\int f(s-t) \delta (s) ds = f(t)$$
%for any function $f$.
Sometimes this is taken as the definition of
Dirac-delta. {\em White Gaussian noise} is a white noise whose
linear filtering is a Gaussian process. This is the usual
definition of white Gaussian noise that can be used in simulating
Gaussian processes. In practice, we use the discrete white
Gaussian noise which is simply $N(0,\sigma^2I_N)$ random
variables. The discrete version of kernel smoothing is
$$\epsilon(t) = \sum_{i=1}^m K(t-\tau_i)w(\tau_i)$$
where $\tau_i$ are regular grid points in $\mathbb{R}^N$. Assume
further that $w(\tau_i) \sim N(0,\sigma^2I_N)$ and $w(\tau)=0$ if
$\tau \neq \tau_i$. Since any linear combination of Gaussian is
again Gaussian, $\epsilon(t_1), \cdots, \epsilon(t_m)$ must be
multivariate normal. Therefore, process $\epsilon(t)$ constructed
in this way must be a Gaussian process.
\begin{verbatim}
kernel=inline('exp(-(x.^2+y.^2)/
                      (2*sigma^2))')
dx=kron(ones(5,1),[-2:2]);
dy=kron(ones(1,5),[-2:2]');
weight=kernel(0.8,dx,dy)/
         sum(sum(kernel(0.8,dx,dy)))
smooth_e=conv2(e, weight,'same');
 y2=g+0.2*smooth_e;
\end{verbatim}

\item The above example shows how to simulate a Gaussian process
in $\mathbb{R}^N$. What if you are supposed to simulate a Gaussian
process in a set $M \in \mathbb{R}^N$. Suppose your observation is
the coordinates of $n$ points that form a curve. In our example
this is the corpus callosum of the human brain. Let
$p_i=(x_i,y_i)$ be the coordinates. We can parameterize the curve
by $t$. Then we can have the following model $$X(t) = \mu_x(t) +
\epsilon_x(t),$$
 $$Y(t) = \mu_y(t) + \epsilon_y(t).$$
where observations are taken at each $t_i$ so that $$x(t_i)=x_i,
y(t_i)=y_i.$$ If you simply assign iid Gaussian noise to the
coordinates $\epsilon_x(t_i) \sim N(0,\sigma_1^2)$ and
$\epsilon_y(t_i) \sim N(0,\sigma_2^2)$, the curve dose not look
very realistic.
\begin{figure}
\centering
\renewcommand{\baselinestretch}{1}
\includegraphics[scale=0.25]{lecture34-1.eps}
\includegraphics[scale=0.25]{lecture34-3.eps}
\includegraphics[scale=0.25]{lecture34-4.eps}
\includegraphics[scale=0.25]{lecture34-5.eps}
\end{figure}
\begin{verbatim}
load CCcurve;
x=CCcurve(:,1);
y=CCcurve(:,2);
ex=0.2*normrnd(0,1,154,1);
ey=0.1*normrnd(0,1,154,1);
plot(x+ex,y+ey)
 \end{verbatim}
So we apply Gaussian kernel smoothing to the iid noises and
generate Gaussian processes or correlated noises
$$\epsilon_x(t) = K*w_x(t), \epsilon_y(t) = K*w_y(t)$$
where $w_x(t_i) \sim N(0,\sigma_1^2)$ and $w_y(t_i) \sim
N(0,\sigma_2^2)$.
\end{enumerate}
\begin{verbatim}
dx=x(2:154)-x(1:153); dx=[ x(1)-x(154) ;dx];
dy=y(2:154)-y(1:153); dy=[ y(1)-y(154) ;dy];
distance = sqrt(dx.^2 + dy.^2)
for j=1:5
    for i=2:153
        nbr_dist=[distance(i-1) 0 distance(i)];
        weight= kernel(nbr_dist,0.2);
        smooth_ex(i) = dot(ex(i-1:i+1),weight);
        smooth_ey(i) = dot(ey(i-1:i+1),weight);
    end;

    nbr_dist=[distance(153) 0 distance(154)];
    weight= kernel(nbr_dist,1);
    nbr_x= [ex(153) ex(154) ex(1)];
    nbr_y= [ey(153) ey(154) ey(1)];
    smooth_ex(154) = dot(nbr_x,weight);
    smooth_ey(154) = dot(nbr_y,weight);
% you put similar code for end point p1
    ex=smooth_ex;
    ey=smooth_ey;
end;
\end{verbatim}
\end{document}
