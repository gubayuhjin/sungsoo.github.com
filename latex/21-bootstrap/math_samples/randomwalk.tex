% This is a simple LaTex sample document that gives a submission format
%   for IEEE PAMI-TC conference submissions.  Use at your own risk.

% Make two column format for LaTex 2e.
\documentclass[12pt]{article} %,twocolumn
%\usepackage[dvips]{graphicx,graphics}
%\usepackage{times,amsmath,amsfonts}

% Use following instead for LaTex 2.09 (may need some other mods as well).
% \documentstyle[times,twocolumn]{article}

% Set dimensions of columns, gap between columns, and paragraph indent
\setlength{\textheight}{9.2in} \setlength{\textwidth}{6.875in}
%\setlength{\columnsep}{0.3125in} \setlength{\topmargin}{0in}
\setlength{\headheight}{0in} \setlength{\headsep}{-1in}
\setlength{\parindent}{1pc}
\setlength{\oddsidemargin}{-.1875in}  % Centers text.
\setlength{\evensidemargin}{-.1875in}

% Add the period after section numbers.  Adjust spacing.
\newcommand{\Section}[1]{\vspace{-8pt}\section{\hskip -1em.~~#1}\vspace{-3pt}}
\newcommand{\SubSection}[1]{\vspace{-3pt}\subsection{\hskip -1em.~~#1}
        \vspace{-3pt}}
\newcommand{\bqn}{\begin{eqnarray}}
\newcommand{\eqn}{\end{eqnarray}}
\newcommand {\diff}[1] {\frac{\partial}{\partial #1}}
\newcommand{\jacob}[3]{\frac{\partial^2 #3}{\partial #1 \partial #2}}
\newcommand{\der}[2]{\frac{\partial #2}{\partial #1}}
\begin{document}

% Make title bold and 14 pt font (Latex default is non-bold, 16pt)
\title{Random Walk}
% For single author (just remove % characters)
\author{Moo K. Chung\\
mchung@stat.wisc.edu}
% For two authors (default example)
\maketitle \thispagestyle{empty} \section*{1D random walk} Suppose
that a particle rests at the origin initially and moves either to
left or to right one step at a time randomly with the probability
of going left $p$ and the probability of going right $q$
($p+q=1$). Determine the probability that the particle is at
position $x$ after $n$ steps.

{\em Solution.} Without loss of generality, we assume $x > 0$. Let
$P_n(x)$ be such probability. It is usually termed the {\em
transition probability}. Obviously $P_n(x) = 0$ if $n < x$ so we
assume $n \geq x$. Also $P_0(x)=\delta_{0,x}$, the Kronecker
delta. We can set up a recurrent relation
$$P_n(x) = qP_{n-1}(x-1) +pP_{n-1}(x+1)$$
and solve it via generating functions to get the transition
probability.

A different approach would as follows. Let $n_L$ be the total
number of steps to the left and $n_R$ be the total number of steps
to the right. Then $x=n_L-n_R$ and $n=n_L + n_R$. So $n_L =
(x+n)/2$. Then $p_n(x)={n \choose n_L}p^{n_L}q^{n-n_L}$, which is
a Binomial distribution with respect to $n_L$ so we have
$\sum_{n_L=0}^n p_n(x) =1$. 
%As you aware from the elementary
%statistic, a binomial distribution can be approximated via a
%normal distribution. We show it now.

%Suppose the particle jumps $\Delta x$ distance to left or right at
%each time step $\Delta t$. Then the recurrent relation is
%$$P_{t+\Delta t}(x) = qP_t(x-\Delta x) + p P_t(x+\Delta x).$$
%Now let $D=2pq(\Delta x)^2/\Delta t$. This constant is called the
%{\em diffusion coefficient} for a reason that will be obvious soon
%(Paul and Baschnagel, 2000). Algebraic manipulation shows
%$$\frac{P_{t+\Delta t}(x) - P_t(x)}{\Delta t} = \frac{q\big(P_t(x-\Delta x)-P_t(x)\big)
%+p\big(P_t(x+\Delta x) - P_t(x)\big)}{\Delta t}$$

\section*{2D random walk}
In the case of 2D lattice grid random walk, we assume that there are 4 directions to move. Then 4 transition probability going one step to east, north, west and south would be $p_1,p_2,p_3,p_4$ respectively. 
\end{document}
