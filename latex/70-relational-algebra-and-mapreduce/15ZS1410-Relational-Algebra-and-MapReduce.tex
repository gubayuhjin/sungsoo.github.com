%%%%%%%%%%%%%%%%%%%%%%%%%%%%%%%%%%%%%%%%%
% Beamer Presentation
% LaTeX Template
% Version 1.0 (10/11/12)
%
% This template has been downloaded from:
% http://www.LaTeXTemplates.com
%
% License:
% CC BY-NC-SA 3.0 (http://creativecommons.org/licenses/by-nc-sa/3.0/)
%
%%%%%%%%%%%%%%%%%%%%%%%%%%%%%%%%%%%%%%%%%

%----------------------------------------------------------------------------------------
%	PACKAGES AND THEMES
%----------------------------------------------------------------------------------------

\documentclass{beamer}

\mode<presentation> {

% The Beamer class comes with a number of default slide themes
% which change the colors and layouts of slides. Below this is a list
% of all the themes, uncomment each in turn to see what they look like.

%\usetheme{default}
%\usetheme{AnnArbor}
%\usetheme{Antibes}
%\usetheme{Bergen}
%\usetheme{Berkeley}
%\usetheme{Berlin}
%\usetheme{Boadilla}
%\usetheme{CambridgeUS}
%\usetheme{Copenhagen}
%\usetheme{Darmstadt}
%\usetheme{Dresden}
\usetheme{Frankfurt}
%\usetheme{Goettingen}
%\usetheme{Hannover}
%\usetheme{Ilmenau}
%\usetheme{JuanLesPins}
%\usetheme{Luebeck}
%\usetheme{Madrid}
%\usetheme{Malmoe}
%\usetheme{Marburg}
%\usetheme{Montpellier}
%\usetheme{PaloAlto}
%\usetheme{Pittsburgh}
%\usetheme{Rochester}
%\usetheme{Singapore}
%\usetheme{Szeged}
%\usetheme{Warsaw}

% As well as themes, the Beamer class has a number of color themes
% for any slide theme. Uncomment each of these in turn to see how it
% changes the colors of your current slide theme.

%\usecolortheme{albatross}
%\usecolortheme{beaver}
%\usecolortheme{beetle}
%\usecolortheme{crane}
%\usecolortheme{dolphin}
%\usecolortheme{dove}
%\usecolortheme{fly}
%\usecolortheme{lily}
%\usecolortheme{orchid}
%\usecolortheme{rose}
%\usecolortheme{seagull}
%\usecolortheme{seahorse}
%\usecolortheme{whale}
%\usecolortheme{wolverine}

%\setbeamertemplate{footline} % To remove the footer line in all slides uncomment this line
%\setbeamertemplate{footline}[page number] % To replace the footer line in all slides with a simple slide count uncomment this line

%\setbeamertemplate{navigation symbols}{} % To remove the navigation symbols from the bottom of all slides uncomment this line
}

\usepackage{times} 
\usepackage{graphicx} % Allows including images
\usepackage{booktabs} % Allows the use of \toprule, \midrule and \bottomrule in tables

% ===============================================================
%    My Commands
    \newcommand{\bi}{\begin{itemize}}
    \newcommand{\ei}{\end{itemize}}
    \newcommand{\be}{\begin{enumerate}}
    \newcommand{\ee}{\end{enumerate}}
    \newcommand{\ii}{\item}
    \newtheorem{Def}{Definition}
    \newtheorem{Lem}{Lemma}

%----------------------------------------------------------------------------------------
%	TITLE PAGE
%----------------------------------------------------------------------------------------

\title[BlinkDB and G-OLA]{Approximate Query Processing in Flying KIWI} % The short title appears at the bottom of every slide, the full title is only on the title page

\author{Sung-Soo Kim} % Your name
\institute[ETRI] % Your institution as it will appear on the bottom of every slide, may be shorthand to save space
{
\textit{sungsoo@etri.re.kr} \\ % Your email address
\medskip
Data Management Research Section, ETRI % Your institution for the title page
}
\date{\today} % Date, can be changed to a custom date

\begin{document}

\begin{frame}
\titlepage % Print the title page as the first slide
\end{frame}

\begin{frame}
\frametitle{References and Slide Credits}
\bi
\ii Jimmy Lin and Chris Dyer, \textit{Data-Intensive Text Processing with MapReduce}, Morgan \& Claypool Publishers, 2010.
\ii Tom White, \textit{Hadoop, The Definitive Guide}, O'Reilly / Yahoo Press, 2012.
\ii Anand Rajaraman, Jeffrey D. Ullman, Jure Leskovec, \textit{Mining of Massive Datasets}, Cambridge University Press, 2013. 
\ei
\end{frame}

\begin{frame}
\frametitle{Outline} % Table of contents slide, comment this block out to remove it
\tableofcontents 
% Throughout your presentation, if you choose to use \section{} and \subsection{} commands, these will automatically be printed on this slide as an overview of your presentation
\end{frame}

%----------------------------------------------------------------------------------------
%	PRESENTATION SLIDES
%----------------------------------------------------------------------------------------


%------------------------------------------------
\section{Introduction} 
% Sections can be created in order to organize your presentation into discrete blocks, all sections and subsections are automatically printed in the table of contents as an overview of the talk
%------------------------------------------------

\begin{frame}
\frametitle{Introduction}
\bi
\ii \textbf{Disclaimer}
\bi
\ii This is not a full course on Relational Algebra
\ii Neither this is a course on SQL
\ei
\ii \textbf{Introduction to Relational Algebra, RDBMS and SQL}
\bi
\ii Follow the video lectures of the Stanford class on RDBMS \href{https://www.coursera.org/course/db}{\texttt{https://www.coursera.org/course/db}}
\ii Note that you have to sign up for an account 
\ei
\ii \textbf{Overview of this part}
\bi
\ii Brief introduction to simplified relational algebra
\ii Useful to understand Pig, Hive and HBase
\ei
\ei
\end{frame}

%------------------------------------------------
\section{Relational Algebra Operators} 
%------------------------------------------------

%------------------------------------------------

\begin{frame}
\frametitle{Relational Algebra Operators}
\bi
\ii \textbf{There are a number of operations on data that fit well the relational algebra model}
\bi
\ii 􏰀In traditional RDBMS, queries involve retrieval of small amounts of data
􏰀\ii 􏰀 In this course, and in particular in this class, we should keep in mind the particular workload underlying MapReduce
\ii 􏰀Full scans of large amounts of data
\ii 􏰀Queries are not selective\footnote{This is true in general. However, most ETL jobs involve selection and projection to do data preparation.}, they process all data
\ei
\ii \textbf{A review of some terminology}
\bi
\ii 􏰀 A \textit{relation} is a table
\ii 􏰀 \textit{Attributes} are the column headers of the table
\ii 􏰀 The set of attributes of a relation is called a \textit{schema}
\ii Example: $R(A_1, A_2, ..., A_n)$ indicates a relation called $R$ whose attributes are $A_1, A_2, ..., A_n$
\ei
\ei
\end{frame}

%------------------------------------------------

%------------------------------------------------
\section{Operators and MapReduce} 
%------------------------------------------------

%------------------------------------------------

\begin{frame}
\frametitle{Bullet Points}

\end{frame}

%------------------------------------------------

%------------------------------------------------
\section{Computing Relational Algebra} 
%------------------------------------------------

%------------------------------------------------

\begin{frame}
\frametitle{Bullet Points}

\end{frame}

%------------------------------------------------


%------------------------------------------------
\section{Summary} 
%------------------------------------------------

%------------------------------------------------

\begin{frame}
\frametitle{Bullet Points}

\end{frame}

%------------------------------------------------



\end{document} 