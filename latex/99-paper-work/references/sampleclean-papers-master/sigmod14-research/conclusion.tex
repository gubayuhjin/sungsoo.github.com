\vspace{-.5em}
\section{Conclusion and Future Work}\label{sec:con}
In this paper, we explore using sampling, integrated with data cleaning, to improve answer quality. We propose \saqpplus, a novel framework which only requires users to clean a sample of data, and utilizes the cleaned sample to obtain unbiased query results with confidence intervals.
We identify three types of data errors (i.e., value error, condition error and duplication error) that may affect query results, and develop \biascorrected and \sampleclean to estimate query results for the data with these errors.
Our analysis and experiments suggest that \saqpplus, which returns the better result between \biascorrected and \sampleclean, is robust to different magnitudes and rates of data errors, and consistently reports good estimate results.
%Moreover, \saqpplus can optimally satisfy user-specified cleaning-cost or result-quality constraints.
Our experiments on both real and synthetic data sets indicate that \saqpplus only needs to clean a small sample of the data to achieve accurate results, and furthermore the size of this sample is independent of the size of the dataset.
In particular, \sampleclean, which processes queries only on the cleaned sample, not only makes the query processing more scalable, but surprisingly may provide higher quality results than an aggregation of the entire dirty data (\alldirty).

To the best of our knowledge, this is the first work to marry data cleaning with sampling-based query processing.
%But, it is merely the tip of the iceberg.
There are many research directions for future exploration. 

\vspace{.25em}
{\noindent \bf Constrained Queries:} Now that we have quantified a tradeoff between cleaning costs and result quality, we can explore query results where users can specify a cost or quality constraint. For example, users may want to know that given a cleaning budget, what is the best result quality they can achieve? Or, given a quality constraint, how many samples they need clean to meet the constraint? 
In these constrained queries, we aim to answer with an optimal cost or most accurate result to meet the constraints.

\vspace{.25em}
{\noindent \bf Uncertain Cleaning Results:}  Our framework can return unbiased query results with respect to AllClean for a variety of different data cleaning approaches.
We are also interested in how we can incorporate uncertain or probabilistic cleaning processes into this framework.
For example, given a dirty record, could the data cleaning module specify a set of ranges for each attribute? 
We are interested in what guarantees, if any, we can achieve in such settings.

\vspace{.25em}
{\noindent \bf Complex SQL Queries:} Another important avenue of future work is to extend our framework to support more complex SQL queries such as join and nested SQL queries. There are some straightforward methods to implement these queries. For example, we can materialize the join result as a single table, and then apply our framework to the materialized table. But this could be very costly for large datasets, thus we need to explore more efficient implementations.
For a larger set of queries, it may not be possible to estimate their results with the CLT.
Thus, exploring empirical estimation approaches (such as bootstrapping) to our framework is another interesting future direction.

\vspace{.25em}
{\noindent \bf Sample Maintenance:} Finally, in real applications users may create multiple samples from the data.
Maintaining these samples for data updates is also very challenging and needs to be investigated.


\vspace{1.5em}

\fussy
{\noindent  \bf Acknowledgements.} {\small  The authors would like to thank Sameer Agarwal, Bill Zhao, and the SIGMOD reviewers for their insightful feedback. This research is supported in part by NSF CISE Expeditions Award CCF-1139158, LBNL Award 7076018, DARPA XData Award FA8750-12-2-0331, the European Research Council under the FP7, ERC MoDaS, agreement 291071, and by the Israel Ministry of Science, and gifts from Amazon Web Services, Google, SAP, The Thomas and Stacey Siebel Foundation, Apple, Inc., Cisco, Cloudera, EMC, Ericsson, Facebook, GameOnTalis, Guavus, Hortonworks, Huawei, Intel, Microsoft, NetApp, Pivotal, Samsung, Splunk, Virdata, VMware, WANdisco and Yahoo!.
}
\sloppy



