% !TEX root = demo.tex
\subsection{Engineering Challenges}
In addition to the numerous research challenges, there have been substantial challenges in engineering \sys.
We highlight two key challenges: API design and Crowdsourcing.

\subsubsection{API Design}
Currently, there are many systems that address specific types of data error \cite{gokhale2014corleone,park2014crowdfill,eracer,chen2014integrating}, but it is often 
the case that data are corrupted in multiple ways.
Unfortunately, there is currently no interoperability between these systems, since some are constraint-based, some use generic crowd APIs, and others have custom crowd interfaces.
In \sys, we address this problem by designing an API that allows for the composition and optimization of data cleaning operators.
Many of the research challenges in \sys rely on clear specifications of the input and output behaviors of
individual data cleaning operators.
For example, Hot Swapping is only possible between two physical operators that have exactly the same input and output specification.
The engineering challenge is to retain modularity and extensibility while restricting operators.
We build a class hierarchy of data cleaning operators which through inheritance and parameterization specify clear input and output behaviors.

One draw back of an extensive API is that the choices may overwhelm a user.
Many of our API methods and classes come with inherited sensible defaults.
For example, if the user fails to completely specify the semantics of the similarity function through the API, 
the system will avoid optimizations and perform a naive Cartesian products and filtering.
These behaviors come naturally through the use of inheritance and polymorphism.

\subsubsection{Crowdsourcing Platform}
The second engineering challenge is designing a platform to serve crowd tasks.
Working with crowds is inherently challenging: unlike with automated operators, the accuracy and speed of processing each tuple varies widely with the crowd worker assigned to it.
We have developed a ``crowd manager" as a layer of indirection to support this complexity.
This component handles serving and processing crowd tasks and automatically uses our techniques for latency reduction as well as state-of-the-art quality control techniques using redundancy and Expectation-Maximization based voting algorithms.

The other benefit of a crowd manager is that it provides an abstraction layer over the varied APIs of Mechanical Turk, Crowd Flower, and others.
We leverage the fact that most crowd platforms serve tasks through a browser-based interface, generating tasks in the form of dynamic web pages and processing responses via AJAX callbacks to our own server.
As a result, tasks submitted to the crowd manager can be processed easily on a variety of different crowds including AMT, Crowd Flower, or a private ``Internal Crowd".