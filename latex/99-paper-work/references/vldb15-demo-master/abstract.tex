% !TEX root = demo.tex
\begin{abstract}
Dealing with dirty data is a fundamental barrier in modern data-driven applications -- 
blindly using results that are derived from dirty data can lead to hidden, yet significant, errors.
To combat dirty data, analysts can easily spend 80\% or more of their analysis 
time~\cite{kandel2012} attempting to identify and understand the data errors, 
while concurrently building custom scripts or large data cleaning workflows 
in order to manage and fix the errors.  Because data cleaning is domain-specific and challenging,
designing such workflows is an iterative process that requires a tight interactive feedback loop.

Existing data cleaning systems are either batch-oriented processing
systems that lack interactivity, or interactive systems that are
designed for a specific data cleaning task (e.g., deduplicating
product data, or finding outliers).  In contrast,
we present \sys, a data cleaning system that distills the core components of existing data cleaning
frameworks into a small set of logical operators that can be composed
into data cleaning plans and incorporates techniques that minimize the latency of the feedback
loop and support dynamic reconfiguration while cleaning plans execute.
We overview the system architecture and these techniques, 
then propose a demonstration designed to showcase how \sys can improve iterative data analysis
and cleaning. The code is available at: \url{http://www.sampleclean.org}.
\end{abstract}

